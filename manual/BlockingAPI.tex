\section{Blocking API}
\label{Sec::BlockingAPI}

Besides event-based actors (the default implementation), \ba also provides context-switching and thread-mapped actors that can make use of the blocking API.
Those actor implementations are intended to ease migration of existing applications or to implement actors that need to have access to blocking receive primitives for other reasons.

Event-based actors differ in receiving messages from context-switching and thread-mapped actors: the former define their behavior as a message handler that is invoked whenever a new messages arrives in the actor's mailbox (by using \lstinline^become^), whereas the latter use an explicit, blocking receive function.

\subsection{Receiving Messages}

The function \lstinline^receive^ sequentially iterates over all elements in the mailbox beginning with the first.
It takes a partial function that is applied to the elements in the mailbox until an element was matched by the partial function.
An actor calling \lstinline^receive^ is blocked until it successfully dequeued a message from its mailbox or an optional timeout occurs.

\begin{lstlisting}
self->receive (
  on<int>().when(_x1 > 0) >> // ...
);
\end{lstlisting}

The code snippet above illustrates the use of \lstinline^receive^.
Note that the partial function passed to \lstinline^receive^ is a temporary object at runtime.
Hence, using receive inside a loop would cause creation of a new partial function on each iteration.
\ba provides three predefined receive loops to provide a more efficient but yet convenient way of defining receive loops.

\begin{tabular*}{\textwidth}{p{0.47\textwidth}|p{0.47\textwidth}}
\lstinline^// DON'T^ & \lstinline^// DO^ \\
\begin{lstlisting}
for (;;) {
  self->receive (
    // ...
  );
}
\end{lstlisting} & %
\begin{lstlisting}
self->receive_loop (
  // ...
);
\end{lstlisting} \\
\begin{lstlisting}
std::vector<int> results;
for (size_t i = 0; i < 10; ++i) {
  self->receive (
    on<int>() >> [&](int value) {
      results.push_back(value);
    }
  );
}
\end{lstlisting} & %
\begin{lstlisting}
std::vector<int> results;
size_t i = 0;
self->receive_for(i, 10) (
  on<int>() >> [&](int value) {
    results.push_back(value);
  }
);
\end{lstlisting} \\
\begin{lstlisting}
size_t received = 0;
do {
  self->receive (
    others() >> [&]() {
      ++received;
    }
  );
} while (received < 10);
\end{lstlisting} & %
\begin{lstlisting}
size_t received = 0;
self->do_receive (
  others() >> [&]() {
    ++received;
  }
).until([&] { return received >= 10; });
\end{lstlisting} \\
\end{tabular*}

\clearpage
\subsection{Receiving Synchronous Responses}

Analogous to \lstinline^sync_send(...).then(...)^ for event-based actors, blocking actors can use \lstinline^sync_send(...).await(...)^.

\begin{lstlisting}
void foo(blocking_actor* self, actor testee) {
  // testee replies with a string to 'get'
  self->sync_send(testee, atom("get")).await(
    [&](const std::string& str) {
      // handle str
    },
    after(std::chrono::seconds(30)) >> [&]() {
      // handle error
    }
  );
}
\end{lstlisting}
